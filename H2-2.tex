\documentclass{article}
\usepackage[cm]{fullpage}
\usepackage{amsmath}
\usepackage{amssymb}
\usepackage{gensymb}
\usepackage{cancel}
\usepackage[utf8]{inputenc}

\title{CS 577 HW 2, P2}
\author{William Jen, Matt Henricks}
\date{}

\begin{document}
\maketitle

\begin{enumerate}
    \item[2.]
        \begin{enumerate}
            \item Given two integers $x$ and $y$, we can denote each integer as a bit array of 
            $\left[x_{n-1}, x_{n-2}, \ldots, x_1, x_0 \right]$ and
            $\left[y_{n-1}, y_{n-2}, \ldots, y_1, y_0 \right]$.
            The traditional approach thus yields the following equation for $xy$:
            
            \begin{equation*}
                xy = \sum\limits_{i = 0}^{n-1} \left(2^i\right)\left(x_{n\ldots0}\right)\left(y_i\right)
            \end{equation*}
            
            Note that the $2^i$ term is $O\left(1\right)$, since it's just a left shift.
            Computing the result of the $\left(x_{n\ldots0}\right)\left(y_i\right)$ term is $O\left(n\right)$
            as you need to perform $n$ multiplications. However, we need to compute $n$ of these, since 
            $i$ goes from $0$ to $n-1$. Thus, building the 2D bit array is $O\left(n^2\right)$.
            
            Let's talk about the summation operation. In the best case with no carries, the number of additions
            we need to do is:
            \begin{equation*}
                1 + 2 + 3 + \ldots + n + \ldots + 3 + 2 + 1
            \end{equation*}
            
            This is starting from the top bit all the way down the zeroth bit. In total, the number of bits to add,
            at a minimum, is $\frac{2\left(n-1\right)\left(n-2\right)}{2} + n = O\left(n^2\right)$. Therefore, since both the 
            creation of the 2D grid and the final summation is $O\left(n^2\right)$, the operation is 
            $O\left(n^2\right)$.
            
            \item If we evaluate each term separately:
                \begin{itemize}
                    \item $x_1y_12^n$ is $T\left(\frac{n}{2}\right)$ since it's only a multiply of $n/2$-digit integers.
                    \item $\left(x_1y_2 + x_2y_1\right)2^{n/2}$ is $2T\left(\frac{n}{2}\right) + c_1n$ since
                        we have 2 $n/2$-digit integer multiplies and 1 addition to recombine.
                    \item $x_2y_2$ is $T\left(\frac{n}{2}\right)$ as it is only a single multiply.
                \end{itemize}
            
                Summing it up, we get a recurrence relation of:
                \begin{equation}
                    T\left(n\right) = 4T\left(\frac{n}{2}\right) + c_1n
                \end{equation}
            
                Next, we use Master's theorem to determine the running time of this algorithm.
                $c_{crit} = \log_2 4 = 2$. $f\left(n\right) = O\left(n\right)$.
                
                Because $c < c_{crit}$, we fall into case 1 and the subproblems dominate runtime.
                Therefore, the running time of this algorithm is $O\left(n^2\right)$.
                
            \item Given the original expansion of:
                \begin{equation*}
                    xy = x_1y_12^n + \left(x_1y_2 + x_2y_1\right)2^{n/2} + x_2y_2
                \end{equation*}
            
                In this equation, the four subproblems we need to compute are: $x_1y_1$, $x_1y_2$,
                $x_2y_1$, and $x_2y_2$. Let's focus on the second group in the original expansion
                and add a couple terms in an attempt to simplify.
                \begin{flalign*}
                    &2^{n/2}\left(x_1y_2 + x_2y_1 + x_1y_1 + x_2y_2 - x_1y_1 - x_2y_2\right) \\
                    &2^{n/2}\left(\left(x_1 + x_2\right)\left(y_1 + y_2\right) - x_1y_1 - x_2y_2\right)
                \end{flalign*}
                
                From this, we recombine into the general expression:
                \begin{equation*}
                    xy = x_1y_12^n + \left(\left(x_1 + x_2\right)\left(y_1 + y_2\right) - x_1y_1 - x_2y_2\right)2^{n/2} + x_2y_2
                \end{equation*}
                
                Now there are only three quantities to compute:
                \begin{itemize}
                    \item $x_1y_1$
                    \item $x_2y_2$
                    \item $\left(x_1 + x_2\right)\left(y_1 + y_2\right)$
                \end{itemize}
                
                Although we incur more addition costs, the subproblems still dominate 
                runtime as per the master theorem.
                This algorithm is $O\left(n^{\log_2 3}\right) = O\left(n^{1.58}\right)$. 
            
        \end{enumerate}
\end{enumerate}
\end{document}
