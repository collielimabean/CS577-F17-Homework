\documentclass{article}
\usepackage[cm]{fullpage}
\usepackage{amsmath}
\usepackage{amssymb}
\usepackage{gensymb}
\usepackage{graphicx}
\usepackage{cancel}
\usepackage{verbatim}
\usepackage[utf8]{inputenc}

\title{CS 577 HW 5, P4}
\author{William Jen, Matt Henricks}
\date{}

\begin{document}
\maketitle

\begin{enumerate}
    \item[4.] 
        \begin{enumerate}
            \item Pseudocode for this algorithm is presented below. 
                \begin{verbatim}
function find_path(n)
    initialize P to be of size [n, n, n + 1] all null values
    
    for i = 1 .. n
        for j = 1 .. n
            P[i, j, 0] = {
                set of { strings from i to itself }* + 
                set of { strings from i directly to j }* +
                set of { strings from j to itself }*
            }
    
    for k = 1 .. n
        for j = k .. n
            for i = k .. n
                P[i, j, k] = P[i, j, k - 1] + (P[i, k, k] + (P[k, k, k]*) + P[k, j, k])
                
    return P[1, n, n] # path     
                \end{verbatim}

            \item \textbf{Explanation:}
                The find\_path function takes in the destination node index $n$, and creates a table
                $P$ for caching computed values. It is a 3D array where $P\left[i, j, k\right]$ represents
                the expression of all paths from start node $i$ to destination node $j$ that goes through
                node of max index $k$. 
                
                If $k = 0$, that means $P\left[i, j, k\right]$ holds the expression of all paths
                directly from node $i$ to $j$, including any self loops from $i$ to itself and $j$ to itself.
                
                We can express the path from $i$ to $j$ that goes through a node with max index $k$ in two ways:
                    \begin{itemize}
                        \item We go through node $k$. If we go through node $k$, we must determine the path
                            from $i$ to $k$, the path from $j$ to $k$, and any self loops that $k$ may have. 
                            The union of these three will result in all possible paths from $i$ to $j$ through
                            node $k$.
                        \item We do not go through node $k$. Thus, we can recursively call find\_path on a 
                            node of max index $k - 1$.
                    \end{itemize}
                
            \item \textbf{Correctness:}
                \begin{enumerate}
                    \item \textbf{Base case:} The base case is $P\left[i, j, 0\right]$, which is the expression
                        of paths directly connecting node $i$ to $j$. As explained above, $P\left[i, j, k\right]$ 
                        includes any self loops from $i$ to itself and $j$ to itself. This is the set of all paths
                        from $i$ to $j$ because of their direct connection.
                        
                    \item \textbf{Inductive Hypothesis:} Suppose we can obtain an expression that contains all
                        possible paths for $P\left[i, j, l\right]$, can we find the same for $P\left[i, j, 
                        l + 1\right]$, for all $0 < i, j < n$?
                        
                    \item \textbf{Inductive Case:} In this $l + 1$ case, we still have two choices: we find
                        a path that goes through the $l + 1$ node, or we do not. If we do not go through the
                        $l + 1$ node, we know by the inductive hypothesis that the path from $i$ to $j$
                        with a max node index of $l$ is expressible. We also know that the union of the
                        expressions of the path from $i$ to $i + 1$, $i + 1$ self-loops, and $i + 1$ to $j$ 
                        is expressible. Because both subproblems are expressible, we can conclude by
                        induction that we can express all possible paths from $i$ to $j$ for any $1 < i, j \leq n$.
                    
                \end{enumerate}
        \end{enumerate}
\end{enumerate}
\end{document}
