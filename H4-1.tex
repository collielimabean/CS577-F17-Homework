\documentclass{article}
\usepackage[cm]{fullpage}
\usepackage{amsmath}
\usepackage{amssymb}
\usepackage{gensymb}
\usepackage{graphicx}
\usepackage{cancel}
\usepackage{verbatim}
\usepackage[utf8]{inputenc}

\title{CS 577 HW 4, P1}
\author{William Jen, Matt Henricks}
\date{}

\begin{document}
\maketitle

\begin{enumerate}
    \item[1.]
        \begin{enumerate}
            \item We have an initial cost $n$ because we have to iterate over $n$ elements for the prefix sum.
                However, this does not take into account the cost of carries. To maximize the number of carries
                in a prefix sum for the base 10 system, we assume the set $\left\{a_1, \ldots, a_n\right\} \in
                \left\{9\right\}$. This produces a carry for 9 out of 10 elements. However, the number of carries
                increases logarithmically with the prefix sum, which also grows linearly with the number of elements.
                This effectively gives a cost function of $c_1n + c_2\log n$, which is $O\left(n\right)$.
                
            \item To "trash" this counter, suppose we have $k < n$ consecutive $1$s to get to $c \cdot 10^i$ where $c, i 
            \in \mathbb{N}$. We then present a sequence of $n - k$ of alternating $\left\{-1, 1\right\}$ to force
            large ripple carries. We repeat the same analysis from part a:
                \begin{itemize}
                    \item We have an initial cost of $k$ to get to $c \cdot 10^i$.
                    \item The remaining $n-k$ elements incur a cost of $n-k + \left(n - k\right)\log k$.
                    \item Adding these two together results in a cost function resembling $n + n\log n$, which is
                        $O\left(n\log n\right)$.
                \end{itemize}
                
            \item Use one counter to add all positive inputs, and the other for all negative inputs.
                After processing all inputs, add them together. This incurs a max of $\log n$ carries. Since
                both counters are monotonically increasing or decreasing, as shown in part a, this is $O\left(n\right)$.
            
            \item
                \begin{verbatim}
function plus_one(cntr):
    carry = 0
    for i = 0 .. (cntr.digit_length - 1):
        if cntr[i] == -1
            cntr[i] = 0 
            carry = 1
        else if cntr[i] == 0
            cntr[i] == 1
            return
        else if cntr[i] == 1
            cntr[i] == -1
            return
            
    cntr.add_top_digit(carry)
            
function minus_one(cntr):
    curr_digit = cntr[0]
    
    # base case #
    if cntr.digit_length == 0
        return
    
    if curr_digit == -1:
        cntr[0] = 1
    else if curr_digit == 1:
        cntr[0] = 0
    else
        # set lowest digit to -1, which is always true in tenary for minus 1.
        # then recursively subtract 1 from the upper digits excluding the last digit.
        minus_one(cntr[1 : cntr.digit_length])
        cntr[0] = -1
                \end{verbatim}
        \end{enumerate}
\end{enumerate}
\end{document}
