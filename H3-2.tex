\documentclass{article}
\usepackage[cm]{fullpage}
\usepackage{amsmath}
\usepackage{amssymb}
\usepackage{gensymb}
\usepackage{graphicx}
\usepackage{cancel}
\usepackage[utf8]{inputenc}
\usepackage{array}
\usepackage{etoolbox}
\usepackage{multicol}
\makeatletter
\patchcmd{\@verbatim}
  {\verbatim@font}
  {\verbatim@font\scriptsize}
  {}{}
\makeatother

\newcommand\undermat[2]{%
  \makebox[0pt][l]{$\smash{\underbrace{\phantom{%
    \begin{matrix}#2\end{matrix}}}_{\text{$#1$}}}$}#2}
    
\title{\large CS 577 HW 3, P2}
\author{\normalsize William Jen, Matt Henricks}
\date{}

\begin{document}
\maketitle

\begin{enumerate}
    \item[2.]
        \begin{enumerate}
            \item A naive algorithm would be to start from the beginning, and count the frequency of that
                element. This results in an $O\left(n^2\right)$ algorithm. To narrow the search space, we find
                the majority elements of the subarrays, and then scan the parent array to determine who is
                the true majority element, if it exists.
            \begin{multicols}{2}
                \begin{verbatim}
function majority(A, start, end):
    # "Base" case - single element only
    if end == start
        return A[1]
        
    # Divide - recur to find majority 
    mid = ceil((end - start) / 2)
    left_majority = majority(A, start, mid)
    right_majority = majority(A, mid + 1, end)
    
    # Same majority element across both
    # including null (absence of majority element)
    if left_majority == right_majority
        return left_majority

    # Find count of left and right majorities compared to "parent" sub-array
    left_majority_count = count_element(A, left_majority, start, end)
    right_majority_count = count_element(A, right_majority, start, end)
    
    # If count of left or right is majority, return. Else null
    if left_majority_count > mid
        return left_majority
    else if right_majority_count > mid
        return right_majority
    else
        return null
            \end{verbatim}
            \columnbreak
            \begin{verbatim}
function count_element(A, element, start, end):
    count = 0    
    if element is null
        return 0
    for i from start -> end
        if A[i] == element
            count += 1
    return count
            \end{verbatim}
        \end{multicols}
            \textbf{Correctness:}
            \begin{itemize}
                \item \textbf{Base case:} For an array $A$ of size 1, the only element is the majority element.
                \item \textbf{Inductive Hypothesis:} If the majority element is found correctly for 
                    any array $A$ with size $\leq k$, is the majority element found correctly for an array
                    with size $k + 1$?
                \item \textbf{Inductive Case:} By the inductive hypothesis, we know that the majority element
                of the left and right subarrays can be found correctly because each subarray size is approximately
                $\frac{k}{2} < k$. If both subarrays have the same majority element, we are done. If they differ, 
                we count the frequency of each majority element in the parent array. If either one of the
                subarray majority elements is a majority element of the parent array, we are done. Otherwise,
                there is no majority element, and we return null.
            \end{itemize}
            
            \textbf{Running Time:} The recurrence relation is $T\left(n\right) = 2T\left(\frac{n}{2}\right) + an$,
            where $a > 0$. The above algorithm splits into two subproblems of size $\frac{n}{2}$ each, and
            performs linear scans when counting the elements for the left and right subarray majority
            elements. As per the Master Theorem, we obtain that $c_{crit} = \log_2 2 = 1$ and $c = 1$ since
            $f\left(n\right) = O\left(n\right)$. Since $c = c_{crit}$, we fall into case two. In our case, $k = 0$
            since $f\left(n\right) = O\left(n\log^0 n\right)$. Therefore, by case 2, this algorithm has a $O\left(n^{c_{crit}}\log^{k+1} n\right) = O\left(n\log n\right)$ runtime.
        
            \item
                \begin{enumerate}
                    \item If $A$ has a majority element $i$, we can visualize the array as if it 
                    were partitioned as such:
                        \[
                            \left[
                                \begin{array}{cccc|cccc}
                                    \undermat{\left\lfloor\frac{n}{2}\right\rfloor + 1}{i & i & \ldots & i} & \undermat{n - \left\lfloor\frac{n}{2}\right\rfloor - 1}{a  & a & \ldots & a} \\
                                \end{array}
                            \right]
                        \]
                        \hfill \break
                        
                        At a minimum, the majority element must have $\left\lfloor\frac{n}{2}\right\rfloor + 1$ elements. A "worst" case scenario is an array with majority element $i$ and the rest containing value $a$. \\
                        
                        The original array size can either be odd or even. Let's look at the two extreme
                        cases when analyzing the even case for how elements can paired:
                        
                        \begin{enumerate}
                            \item[1)]  Pair each majority element $i$ with an element $a$. \\
                                Pairing an $i$ with an $a$ discards $\frac{n}{2} - 1$ elements, since there 
                                are $\frac{n}{2} - 1$ $a$ elements. This results in two $i$ elements remaining.
                                These reduce into a single $i$ element, which is a majority element.
                            \item[2)] Pair each majority element $i$ element together, and each $a$ element together. \\
                                Pairing $i$ elements together and $a$ elements together results in a reduced
                                array with $\left\lceil\frac{n}{4}\right\rceil$ $i$ elements and
                                $\left\lceil\frac{n}{4}\right\rceil - 1$ $a$ elements. 
                                Since $\frac{n}{4} > \frac{n}{4} - 1$, $i$ is still a majority element. 
                                In total, there are $\frac{n}{4} + \frac{n}{4} - 1 = \frac{n}{2} - 1$ 
                                elements in the reduced array. Because $\frac{n}{4} > \frac{1}{2}(\frac{n}{2} -
                                1)$, $i$ is  a majority element for this reduced array.
                        \end{enumerate}
                        
                        \hfill \break
                        \textbf{\textit{These two cases of extremes cover all possible pairings for an even
                        length array.}} For each majority element $i$, we can either pair it 
                        with another element $i$ or an element $a$. Since $i$ has greater than or equal to 
                        half of the $n$ elements ($\frac{n}{2} + 1$ elements), 
                        if all $n$ elements are grouped into $\frac{n}{2}$ groups of size $2$, by Pigeonhole 
                        Principle you are guaranteed to have at least one group which contains two $i$ elements.
                        
                        \hfill \break
                        Removing these two $i$ elements, we are left with $n - 2$ elements to distribute to 
                        $\frac{n}{2} - 1$ groups of size $2$. There are now $\frac{n}{2} + 1 - 2 =  \frac{n}{2} - 1$
                        $i$ elements and $\frac{n}{2} - 1$ $a$ elements. There is no possible arrangement of these 
                        existing elements such that the pairs (two $i$, two $a$, or one $i$s and $a$s) 
                        "cancel out" the guaranteed pair of $i$ elements. 
                        
                        \hfill \break
                        For odd cases, the professor noted that the reduction will result in
                        a single element remaining, and that there exists a way to have the majority
                        element $i$ not be a majority element in the reduced array whether you drop the
                        singleton or not. To remedy this, we select a random element $x$. 
                        We scan the array to determine if $x$ is a majority element. 
                        If $x$ is a majority element, then we are done - we found it in linear time.
                        If not, we throw it away. We now have an array with length $n-1$, which is even. 
                        For the minimum (worst) original odd-length array with a majority element $i$, there is exactly
                        one more element $i$ than other elements. By discarding a non-majority element, 
                        we get an even length array where there are exactly two more majority elements $i$
                        than other elements. This case thus degenerates into the above even-length array
                        situation, which has been proved to always result in the majority element $i$ being
                        the majority element in the reduced array.
                        
                        \hfill \break
                        Dropping a non-majority element in the odd case does not affect correctness in any
                        way. In an odd-length array, there are $\left\lfloor\frac{n}{2}\right\rfloor + 1$ 
                        majority elements and $\left\lfloor\frac{n}{2}\right\rfloor$ non-majority elements.
                        If we remove a non-majority element, we can rewrite the 
                        number of majority elements to be $\frac{n}{2} + 1$. Because $\frac{n}{2} + 1 >
                        \frac{1}{2}\left(n - 1\right)$, the majority element is still the majority
                        element after dropping it from the original odd-length array.
                        
                        \hfill \break
                        Therefore, regardless of the grouping strategy, the majority element $i$ 
                        will be the majority element of the reduced array.
                        
                    \item If the reduced array has a majority element, does the array $A$ have the 
                        same majority element? No. For example, the array $\left[1, 1, 2, 3, 4, 5\right]$ can reduce into $\left[1\right]$. 1 is a majority element in the reduced
                        array, but is not a majority element in $A$.
                        
                    \item 
                        \begin{enumerate}
                            \item[1)] First, check if the array size is odd. If not, go to (2).
                                If it is, pick the first element, and determine if it is a majority element. 
                                Do this by initializing a counter to zero, and scan the array for matches. 
                                If the counter value is greater  than $\left\lfloor\frac{n}{2}\right\rfloor$, 
                                we are done. If not, discard it to make the array even length.
                            \item[2)] Now that the array has an even length, we simply pair elements and
                                if they are the same, keep one. If they differ, toss them both. This
                                reduces the array size by at most half. Repeat the first and second steps
                                until the size is 1.
                            \item[3)] The majority element will be remaining element when the array
                                is reduced to size 1.
                        \end{enumerate}
                        
                        \hfill \break
                        The first step incurs a cost of $n$ steps. We then perform $\frac{n}{2}$ steps
                        to reduce the array, and further reduce by half each step. This results in a 
                        recurrence relation of:
                        \begin{equation*}
                            T\left(n\right) = T\left(\frac{n}{2}\right) + an
                        \end{equation*}
                        
                        where $a > 0$. By the Master Theorem, $c_{crit} = \log_21 = 0$ and $c = 1$.
                        Since $c_{crit} < c$, we fall into case 1 where the $an$ term bounds the runtime. 
                        Since $ f\left(n\right)= an = O\left(n\right)$, this algorithm is $O\left(n\right)$.
                \end{enumerate}
        \end{enumerate}
\end{enumerate}
\end{document}
    