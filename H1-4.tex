\documentclass{article}
\usepackage[cm]{fullpage}
\usepackage{amsmath}
\usepackage{amssymb}
\usepackage{gensymb}
\usepackage{cancel}
\usepackage[utf8]{inputenc}

\title{CS 577 HW 1, P4}
\author{William Jen, Matt Henricks}
\date{}

\begin{document}
\maketitle

\begin{enumerate}
    \item[4]
        \begin{enumerate}
            \item Let $d_n$ where n $\epsilon \{ 0, 1, 2 \}$ denote a disc at each tower.
            Each disc has a unique size, or in other words, $d_0 \neq d_1 \neq d_2$. Therefore, there
            exists a tower with the smallest disc, let this be denoted by $d_n$. Hence, 
            $d_{\left(n + 1\right) \bmod 3}$ must be larger than $d_n$, which means
            that there will always be at least one legal move for any legal configuration.
            
            \item 
                \textbf{Base case}: $n = 1$ \\
                    Two moves are required to move the disc from tower 0 to tower 2. Two is finite.
                    
                \textbf{Inductive case}: Suppose $T(n)$ is finite. Does $T(n+1)$ hold?\\
                If we want to move $n + 1$ discs from tower 0 to tower 2, we can break it down into two parts,
                a group containing disc of size $1 \ldots n$ and the disc of size $n + 1$. 
                The following moveset solves the modified Towers of Hanoi. Let $T(n)$ be the number of moves 
                required to move $n$ discs from tower 0 to tower 2, and $S(n)$ be the number of moves required
                to move $n$ discs from tower 0 to tower 1. 
                \begin{itemize}
                    \item Move the $n$ discs to tower 2, which requires $T(n)$ moves.
                    \item Move the $n+1$ disc to tower 1.
                    \item Move the $n$ discs to tower 0, which requires $S(n)$ moves.
                    \item Move the $n+1$ disc to tower 2.
                    \item Move the $n$ discs to tower 2, which requires $T(n)$ moves.
                \end{itemize}
                
                Because $T(n+1)$ is a sum of all finite quantities, it is therefore finite.
                
            \item 
                From part (b), the required number of moves to move $n$ discs from tower 0 to tower 2 is:
                \begin{equation}
                    T(n) = 2T(n-1) + S(n-1) + 2
                \end{equation}
                
                In order to fully resolve equation (1), we must determine the recurrence relation for
                $S(n)$. The fewest number of moves to transfer $n$ discs from tower 0 to 1 is:
                \begin{itemize}
                    \item Move the $n-1$ discs from tower 0 to tower 2, which requires $T(n-1)$ moves.
                    \item Move the $n$ disc from tower 0 to tower 1.
                    \item Move the $n-1$ discs from tower 2 to tower 1, which requires $T(n-1)$ moves.
                \end{itemize}
                
                We therefore derive the following recurrence relation for $S(n)$:
                \begin{equation}
                    S(n) = 2T(n-1) + 1
                \end{equation}
                
                Combining the recurrence relations found in (1) and (2), we get:
                \begin{equation}
                    \boxed{T(n) = 2T(n-1) + 2T(n-2) + 3}
                \end{equation}
                
            \item 
                \textbf{Base Case}: $n = 1$
                \begin{flalign*}
                    &T(1) = 2 \\
                    &T(1) \stackrel{?}{\geq} \left(\sqrt{3} + 1\right)^{1 - 1} \\
                    &2 \geq 1 \hspace{5pt} \checkmark
                \end{flalign*}
                
                \textbf{Inductive Case}: Suppose $T(k) \geq \left(\sqrt{3} + 1\right)^{k - 1}$
                for some $k > 1$. Does $T(k + 1)$ hold? 
                \begin{flalign*}
                    &T(k + 1) = 2T((k + 1) - 1) + 2T((k + 1) - 2) + 3 \stackrel{?}{\geq} \left(\sqrt{3} + 1\right)^{k}\\
                    &2\left(\sqrt{3} + 1\right)^{k - 1} + 2\left(\sqrt{3} + 1\right)^{k - 2} + 3 \stackrel{?}{\geq} \left(\sqrt{3} + 1\right)^{k}  \\
                    &2\left(\sqrt{3} + 1\right)^{k - 2}\left[\sqrt{3} + 1 + 1\right] + 3 \stackrel{?}{\geq} \left(\sqrt{3} + 1\right)^{k}  \\
                    &\left(2\sqrt{3} + 4\right)\left(\sqrt{3} + 1\right)^{k - 2} + 3 \stackrel{?}{\geq} \left(\sqrt{3} + 1\right)^{k} \\
                    &3 \stackrel{?}{\geq} \left(\sqrt{3} + 1\right)^{k} - \left(2\sqrt{3} + 4\right)\left(\sqrt{3} + 1\right)^{k - 2} \\
                    &3 \stackrel{?}{\geq} \left(\sqrt{3} + 1\right)^{k - 2} \cancelto{0}{\left(\left(\sqrt{3} + 1\right)^2 - \left(2\sqrt{3} + 4\right)\right)} \\
                    &3 \geq 0 \hspace{5pt} \checkmark
                \end{flalign*}
                
                We can therefore conclude by induction that $T(n) \geq \left(\sqrt{3} + 1\right)^{n - 1}$.
                
            \item We can use the annihilator method to determine the upper bound. Given the recurrence relation
            in (3), we rearrange and simplify as follows: 
                \begin{flalign*}
                    &T(n) - 2T(n-1) - 2T(n-2) = 3 \\
                    &(\textbf{E}^2 - 2\textbf{E} - 2)(\textbf{E} - 3) = 0 \\
                    &(\textbf{E} - (1 + \sqrt{3}))(\textbf{E} - (1 - \sqrt{3}))(\textbf{E} - 3) = 0
                 \end{flalign*}
            
                Converting from the annihilator back to a regular function, we get:
                \begin{equation*}
                    T(n) = \alpha 3^n + \beta (1 + \sqrt{3})^n + \delta (1 - \sqrt{3})^n \text{ for some } \alpha, \beta, \delta \hspace{5pt} \epsilon \hspace{5pt} \mathbb{R}
                \end{equation*}
            
                A guess for an upper bound for $T(n)$ could be $3^n$, since that is the largest growing term.
                
                \textbf{Base Case}: $n = 1$
                \begin{flalign*}
                    &T(1) = 2 \\
                    &T(1) \stackrel{?}{\leq} 3^n \\
                    &2 \leq 3 \hspace{5pt} \checkmark
                \end{flalign*}
                
                \textbf{Inductive Case}: Suppose $T(k) \leq 3^k$
                for some $k > 1$. Does $T(k + 1)$ hold? 
                \begin{flalign*}
                    &T(k + 1) \stackrel{?}{\leq} 3^{k+1} \\
                    &T(k + 1) = 2T(k) + 2T(k-1) + 3 \\
                    &2\left(3^k\right) + 2\left(3^{k-1}\right) + 3 \stackrel{?}{\leq}  3^{k+1} \\
                    &2\left(3^k + 3^{k-1}\right) + 3 \stackrel{?}{\leq} 3^{k+1} \\
                    &2\left(3^{k-1}\right)\left(3 + 1\right) + 3 \stackrel{?}{\leq} 3^{k+1}\\
                \end{flalign*}
                
                We finally arrive at the following inequality:
                \begin{equation}
                    8\left(3^{k-1}\right) + 3 \stackrel{?}{\leq}  9\left(3^{k-1}\right) 
                \end{equation}
                
                If we solve for the condition where both sides are equal, we find that Equation (4) holds for all $k \ge 2$.
                
                Since $k = 1$ was proven true in the base case, we have proven that $3^n$
                is an upper bound for $T(n)$ for all $ k > 0$.
        \end{enumerate}
\end{enumerate}
\end{document}
 