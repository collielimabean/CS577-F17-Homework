\documentclass{article}
\usepackage[cm]{fullpage}
\usepackage{amsmath}
\usepackage{amssymb}
\usepackage{gensymb}
\usepackage{graphicx}
\usepackage{cancel}
\usepackage{verbatim}
\usepackage[utf8]{inputenc}

\title{CS 577 HW 5, P1}
\author{William Jen, Matt Henricks}
\date{}

\begin{document}
\maketitle

\begin{enumerate}
    \item[1.]
        \begin{enumerate}
            \item
                Each item has a size $s_i$ and value $v_i$ where $i = \left\{1, \ldots, n\right\}$.
                \begin{verbatim}
function maximize_revenue(T, n, S, v, s, k)
    # T is the cache table
    # n is the number of items - it represents the index within the array of items
    # S is the maximum capacity of the burglar's car
    # v, s are the value and size arrays
    # k tells us if we chose the current item or not
    
    if len(n) == 0
        return 0
    
    # we've already computed this step - return it
    if T[n][S] != null
        return T[n][S]
    
    # this item can't fit - ignore it
    if s[n] > S
        return maximize_revenue(T, n - 1, S, v, s)
    
    # two choices: we take it or we dont
    # compute both, and see which is better
    select_this_item = v[n] + maximize_revenue(T, n - 1, S - s[n])
    dont_select_this_item = maximize_revenue(T, n - 1, S)
    
    # cache value for later computations
    if select_this_item > dont_select_this_item
        T[n][S] = select_this_item
        k[n][S] = true
    else
        T[n][S] = dont_select_this_item
        k[n][S] = false
    
    return T[n][S]
    
# call maximize function
max_money = maximize_revenue(T, n, S, v, s, k)

# output ideal path now that k is populated
s_prime = S
for n_prime = n .. 1
    if k[n_prime][s_prime] == true
        print n_prime # take this item
        s_prime = s_prime - s[n_prime]
                \end{verbatim}

            \item The runtime of this algorithm is $O\left(nS\right)$. Our table is a 2D array where the row
                is of size $n$ and the columns are of size $S$. Although there are two recursive calls per function call,
                the table prevents repeating work, thus effectively enforcing one recursive call for each unique item and
                capacity value.
                
            \item \textbf{Correctness}:
                \begin{enumerate}
                    \item \textbf{Base Case:} $S = 0$. Regardless of $n$, the maximum value is 0.
                    \item \textbf{Inductive Hypothesis:} If we can determine an optimal group of items to take
                        for a capacity $S$, can we find the optimal group of items for $S + 1$?
                    \item \textbf{Inductive Case:}
                        For a capacity $S + 1 > S$, the algorithm computes and takes the maximum value
                        to either select or not select item $k \leq n$. Because this computation itself
                        relies on subproblems which are known to be optimal selections through the induction
                        hypothesis, we know this to be optimal.
                \end{enumerate}
            
        \end{enumerate}
\end{enumerate}
\end{document}
