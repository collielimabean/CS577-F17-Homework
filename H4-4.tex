\documentclass{article}
\usepackage[cm]{fullpage}
\usepackage{amsmath}
\usepackage{amssymb}
\usepackage{gensymb}
\usepackage{graphicx}
\usepackage{cancel}
\usepackage[utf8]{inputenc}

\title{CS 577 HW 4, P4}
\author{William Jen, Matt Henricks}
\date{}

\begin{document}
\maketitle

\begin{enumerate}
    \item[4.]
        \begin{enumerate}
            \item The idea is to find the set of all overlapping intervals and all disjoint intervals. 
                For any overlapping interval, only keep the overlapping part. Always keep the disjoint intervals. Continually refine until the remaining intervals are disjoint. From there, pick a point from 
                each set to construct
                $S$. 
                \begin{enumerate}
                    \item First, construct an interval tree, which is defined in CLRS 14.3. It is a red-black
                        tree where each node represents a closed interval. In order to compare nodes
                        against each other, we select the starting point of the interval as the key.
                    \item Traverse to the smallest start point by continually taking the left branch on the
                        interval tree. Initialize an additional empty interval tree.
                    \item Begin an in-order traversal and pairwise-compare each node. Specifically, for each 
                        node (which represents an interval), look ahead to the next node in the 
                        in-order traversal, and add the intersecting interval of these two nodes to the 
                        new interval tree, if any. If these nodes are disjoint, simply add both nodes
                        (intervals) to the new interval tree. 
                    \item Repeat this process on newly generated interval trees until the generated
                        interval tree contains only disjoint intervals. Pick a point from each interval, 
                        and you obtain the minimum set $S$.
                \end{enumerate}
            
            \item
                \begin{enumerate}
                    \item Constructing an interval tree is $O\left(n \log n\right)$, because balancing
                        the tree is $\log n$, and we have $n$ elements to insert.
                    \item Traversing to the smallest element takes $O\left(\log n\right)$ time.
                    \item Node comparisons for intersection against each other is $O\left(1\right)$. The in-order
                        traversal runs in $O\left(n\right)$ time, with insertion into the new interval tree
                        running in $O\left(\log n\right)$ time.
                    \item The number of iterations required to arrive at the minimal set $S$ strongly depends
                        on the input set of intervals. For example, if the original set of intervals is completely
                        disjoint, the best case runtime for this algorithm is $O\left(n\log n\right)$ because
                        of the cost of constructing the interval tree. On the
                        other hand, if there are many intersections between intervals, then the number of 
                        iterations required to get to a disjoint set of intervals is higher. On average,
                        it is reasonable to choose $\log n$ iterations to arrive at the minimal set $S$.
                \end{enumerate}
                
                Therefore, on average, the runtime for this algorithm is $O\left(\left(\log n\right) 
                n\log n\right) = O\left(n\log^2 n\right)$.
                
            \item
                \begin{enumerate}
                    \item \textbf{Base Case:} A single interval is disjoint, therefore selecting a point
                        from that single interval is the minimum set $S$.
                    \item \textbf{Inductive Hypothesis: } Suppose this algorithm correctly generates
                        the minimum set $S$ for $k$ intervals. Does this algorithm produce correct
                        output for $k + 1$ intervals?
                    \item \textbf{Inductive Case: }
                        After the first reduction pass, there are two outcomes: either the $k + 1$ 
                        intervals reduce into $k$ or fewer intervals, or all intervals are disjoint. From
                        the inductive hypothesis, we know we can find the minimum set $S$ from the $k$ or
                        fewer intervals. If all intervals are disjoint, we can choose any point from each
                        interval to produce $S$.
                \end{enumerate}
        \end{enumerate}
\end{enumerate}
\end{document}